\documentclass[journal,12pt,twocolumn]{IEEEtran}

\usepackage{setspace}
\usepackage{gensymb}

\singlespacing


\usepackage[cmex10]{amsmath}

\usepackage{amsthm}

\usepackage{mathrsfs}
\usepackage{txfonts}
\usepackage{stfloats}
\usepackage{bm}
\usepackage{cite}
\usepackage{cases}
\usepackage{subfig}

\usepackage{longtable}
\usepackage{multirow}

\usepackage{enumitem}
\usepackage{mathtools}
\usepackage{steinmetz}
\usepackage{tikz}
\usepackage{circuitikz}
\usepackage{verbatim}
\usepackage{tfrupee}
\usepackage[breaklinks=true]{hyperref}
\usepackage{graphicx}
\usepackage{graphics}
\usepackage{tkz-euclide}
\usepackage{float}

\usetikzlibrary{calc,math}
\usepackage{listings}
    \usepackage{color}                                            %%
    \usepackage{array}                                            %%
    \usepackage{longtable}                                        %%
    \usepackage{calc}                                             %%
    \usepackage{multirow}                                         %%
    \usepackage{hhline}                                           %%
    \usepackage{ifthen}                                           %%
    \usepackage{lscape}     
\usepackage{multicol}
\usepackage{chngcntr}

\DeclareMathOperator*{\Res}{Res}

\renewcommand\thesection{\arabic{section}}
\renewcommand\thesubsection{\thesection.\arabic{subsection}}
\renewcommand\thesubsubsection{\thesubsection.\arabic{subsubsection}}

\renewcommand\thesectiondis{\arabic{section}}
\renewcommand\thesubsectiondis{\thesectiondis.\arabic{subsection}}
\renewcommand\thesubsubsectiondis{\thesubsectiondis.\arabic{subsubsection}}


\hyphenation{op-tical net-works semi-conduc-tor}
\def\inputGnumericTable{}                                 %%

\lstset{
%language=C,
frame=single, 
breaklines=true,
columns=fullflexible
}
\begin{document}


\newtheorem{theorem}{Theorem}[section]
\newtheorem{problem}{Problem}
\newtheorem{proposition}{Proposition}[section]
\newtheorem{lemma}{Lemma}[section]
\newtheorem{corollary}[theorem]{Corollary}
\newtheorem{example}{Example}[section]
\newtheorem{definition}[problem]{Definition}

\newcommand{\BEQA}{\begin{eqnarray}}
\newcommand{\EEQA}{\end{eqnarray}}
\newcommand{\define}{\stackrel{\triangle}{=}}
\newcommand\hlight[1]{\tikz[overlay, remember picture,baseline=-\the\dimexpr\fontdimen22\textfont2\relax]\node[rectangle,fill=blue!50,rounded corners,fill opacity = 0.2,draw,thick,text opacity =1] {$#1$};}
\bibliographystyle{IEEEtran}
\providecommand{\mbf}{\mathbf}
\providecommand{\pr}[1]{\ensuremath{\Pr\left(#1\right)}}
\providecommand{\qfunc}[1]{\ensuremath{Q\left(#1\right)}}
\providecommand{\sbrak}[1]{\ensuremath{{}\left[#1\right]}}
\providecommand{\lsbrak}[1]{\ensuremath{{}\left[#1\right.}}
\providecommand{\rsbrak}[1]{\ensuremath{{}\left.#1\right]}}
\providecommand{\brak}[1]{\ensuremath{\left(#1\right)}}
\providecommand{\lbrak}[1]{\ensuremath{\left(#1\right.}}
\providecommand{\rbrak}[1]{\ensuremath{\left.#1\right)}}
\providecommand{\cbrak}[1]{\ensuremath{\left\{#1\right\}}}
\providecommand{\lcbrak}[1]{\ensuremath{\left\{#1\right.}}
\providecommand{\rcbrak}[1]{\ensuremath{\left.#1\right\}}}
\theoremstyle{remark}
\newtheorem{rem}{Remark}
\newcommand{\sgn}{\mathop{\mathrm{sgn}}}
\providecommand{\abs}[1]{\left\vert#1\right\vert}
\providecommand{\res}[1]{\Res\displaylimits_{#1}} 
\providecommand{\norm}[1]{\left\lVert#1\right\rVert}
%\providecommand{\norm}[1]{\lVert#1\rVert}
\providecommand{\mtx}[1]{\mathbf{#1}}
\providecommand{\mean}[1]{E\left[ #1 \right]}
\providecommand{\fourier}{\overset{\mathcal{F}}{ \rightleftharpoons}}
%\providecommand{\hilbert}{\overset{\mathcal{H}}{ \rightleftharpoons}}
\providecommand{\system}{\overset{\mathcal{H}}{ \longleftrightarrow}}
	%\newcommand{\solution}[2]{\textbf{Solution:}{#1}}
\newcommand{\solution}{\noindent \textbf{Solution: }}
\newcommand{\cosec}{\,\text{cosec}\,}
\providecommand{\dec}[2]{\ensuremath{\overset{#1}{\underset{#2}{\gtrless}}}}
\newcommand{\myvec}[1]{\ensuremath{\begin{pmatrix}#1\end{pmatrix}}}
\newcommand{\mydet}[1]{\ensuremath{\begin{vmatrix}#1\end{vmatrix}}}
\numberwithin{equation}{subsection}
\makeatletter
\@addtoreset{figure}{problem}
\makeatother
\let\StandardTheFigure\thefigure
\let\vec\mathbf
\renewcommand{\thefigure}{\theproblem}
\def\putbox#1#2#3{\makebox[0in][l]{\makebox[#1][l]{}\raisebox{\baselineskip}[0in][0in]{\raisebox{#2}[0in][0in]{#3}}}}
     \def\rightbox#1{\makebox[0in][r]{#1}}
     \def\centbox#1{\makebox[0in]{#1}}
     \def\topbox#1{\raisebox{-\baselineskip}[0in][0in]{#1}}
     \def\midbox#1{\raisebox{-0.5\baselineskip}[0in][0in]{#1}}
\vspace{3cm}
\title{Assignment-9}
\author{T.Naveena}
\maketitle
\newpage
\bigskip
\renewcommand{\thefigure}{\theenumi}
\renewcommand{\thetable}{\theenumi}
Download all python codes from 
\begin{lstlisting}
https://github.com/ThurpuNaveena/Matrix-Theory/tree/main/Assignment9/Codes
\end{lstlisting}
%
and latex-tikz codes from 
%
\begin{lstlisting}
https://github.com/ThurpuNaveena/Matrix-Theory/tree/main/Assignment9
\end{lstlisting}
%
\section{Question No. 2.14}
A factory makes tennis rackets and cricket bats. A tennis racket takes 1.5 hours
of machine time and 3 hours of craftman’s time in its making while a cricket bat
takes 3 hour of machine time and 1 hour of craftman’s time. In a day, the factory
has the availability of not more than 42 hours of machine time and 24 hours of
craftsman’s time.\\
(i) What number of rackets and bats must be made if the factory is to work
at full capacity?\\
(ii)If the profit on a racket and on a bat is Rs 20 and Rs 10 respectively, find
the maximum profit of the factory when it works at full capacity.
\section{Solution}
\numberwithin{table}{section}
\begin{table}[!ht]
\centering
\resizebox{\columnwidth}{!}{\begin{tabular}{|c|c|c|c|} 
\hline
item & Machine hours & Craftman's hours & profit \\
\hline
Tennis Racket & 1.5 & 3  &  20  \\ 
\hline
Cricket Bats & 3 & 1 &  10  \\ 
\hline
Maximum time Available & 42 & 24 & \\ 
\hline
\end{tabular}}
\caption{factory Requirements}
\label{tab:table1}
\end{table}
Let the number of Tennis Rackets  be $x$ and the number of cricket bats be $y$  such that 
\begin{align}
x \geq 0 \\
y \geq 0 
\end{align}
According to the question,
\begin{align}
1.5x+3y &\leq 42 \\
\implies 3x+6y &\leq 84 \\
\implies x+2y &\leq 28 
\end{align}
and,
\begin{align}
3x+y &\leq 24 
\end{align}
$\therefore$ Our problem is
\begin{align}
\max_{\vec{x}} Z &= \myvec{20 & 10}\vec{x}\\
s.t. \quad \myvec{1 & 2 \\ 3 & 1}\vec{x} &\preceq \myvec{28\\24} 
\end{align}
Lagrangian function is given by
\begin{equation}
\begin{aligned}
&L(\vec{x},\boldsymbol{\lambda}) \\ &= \myvec{20 & 10}\vec{x}+\lcbrak{\sbrak{\myvec{1 & 2}\vec{x}-28}} \\ &+ \sbrak{\myvec{3 & 1}\vec{x}-24}\\ &+ \sbrak{\myvec{-1 & 0}\vec{x}} +\rcbrak{\sbrak{\myvec{0 & -1}\vec{x}}}\boldsymbol{\lambda}
\end{aligned}
\end{equation}
where,
\begin{align}
\boldsymbol{\lambda} &= \myvec{\lambda_1 \\ \lambda_2 \\ \lambda_3 \\ \lambda_4 \\ \lambda_5 \\ \lambda_6}
\end{align}
Now,
\begin{align}
\nabla L(\vec{x},\boldsymbol{\lambda}) &= \myvec{20+ \myvec{1 & 3  & -1 & 0 }\boldsymbol{\lambda}\\ 10+\myvec{2 & 1 & 0 & -1}\boldsymbol{\lambda} \\ \myvec{1 & 2}\vec{x}-28 \\ \myvec{3 & 1}\vec{x}-24 \\  \myvec{-1 & 0}\vec{x} \\ \myvec{0 & -1}\vec{x}}
\end{align}
$\therefore$ Lagrangian matrix is given by
\begin{align}
\myvec{0 & 0 & 1 & 3 & -1 & 0 \\ 0 & 0 & 2 & 1  & 0 & -1 \\ 1 & 2 & 0 & 0 & 0 & 0 \\ 3 & 1 & 0 & 0 & 0 & 0  \\ -1 & 0 & 0 & 0 & 0 & 0  \\ 0 & -1 & 0 & 0 & 0 & 0 }\myvec{\vec{x} \\ \boldsymbol{\lambda} } &= \myvec{-20 \\ -10 \\ 28 \\ 24 \\ 0 \\0 }
\end{align}
Considering $\lambda_1,\lambda_2$ as only active multiplier,
\begin{align}
\myvec{0 & 0 & 1 & 3 \\ 0 & 0 & 2 & 1 \\ 1 & 2 & 0 & 0 \\ 3 & 1 & 0 & 0}\myvec{\vec{x}\\ \boldsymbol{\lambda}} &= \myvec{-20 \\ -10 \\ 28 \\ 24}
\end{align}
resulting in,
\begin{align}
\myvec{\vec{x} \\ \boldsymbol{\lambda}} &= \myvec{0 & 0 & 1 & 3 \\ 0 & 0 & 2 & 1 \\ 1 & 2 & 0 & 0 \\ 3 & 1 & 0 & 0}^{-1}\myvec{-20 \\ -10 \\ 28 \\ 24}
\\
\implies   \myvec{\vec{x} \\ \boldsymbol{\lambda}} &= \myvec{0 & 0 & \frac{-1}{5} & \frac{2}{5} \\ 0 & 0 & \frac{3}{5} & \frac{-1}{5} \\ \frac{-1}{5} & \frac{3}{5} & 0 & 0 \\ \frac{2}{5} & \frac{-1}{5} & 0 & 0}\myvec{-20 \\ -10 \\ 28 \\ 24}
\\
\implies \myvec{\vec{x} \\ \boldsymbol{\lambda}} &= \myvec{4 \\ 12 \\ -2 \\ -6 }
\end{align}
$\because \boldsymbol{\lambda}=\myvec{-2 \\ -6} \succ \vec{0} $
\\
$\therefore$ Optimal solution is given by
\begin{align}
    \vec{x} &= \myvec{4\\12} \\
    Z &= \myvec{20 & 10}\vec{x} \\
    &= \myvec{20 & 10}\myvec{4 \\ 12} \\
    &= 200
\end{align}
By using cvxpy in python ,
\begin{align}
    \vec{x}=\myvec{3.99999998\\12.0000000}\\
    Z = 199.99999964
\end{align}
Hence ,\boxed{x=4} Tennis Rackets and \boxed{y=12} Cricket Bats should be used to maximum time Available profit \boxed{Z=200}.\\
Thus,
(i) 4 Tennis Rackets and 12 Cricket Bats must be made so that factory runs at full capacity.\\
(ii) Maximum profit is Rs 200, When 4 Tennis Bats and 12 Cricket Bats are produced.
\numberwithin{figure}{section}
\begin{figure}[!ht]
\centering
\includegraphics[width=\columnwidth]{Figure.png}
\caption{Graphical Solution}
\label{fig: Graphical Solution}	
\end{figure}
\end{document}
